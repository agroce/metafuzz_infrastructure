\section{Current and Pending}
\cut{
(ii) Current and Pending (Other) Support
Note: The mandate to use SciENcv only for the preparation of Current and Pending (Other) Support information will go into effect for new proposals submitted or due on or after October 23, 2023. In the interim, proposers may continue to prepare and submit this document via use of SciENcv or the NSF fillable PDF. NSF, however, encourages the community to use SciENcv prior to the October 2023 implementation.

(a) Current and Pending (Other) Support[30] information is used to assess the capacity of the individual to carry out the research as proposed and helps assess any potential scientific and budgetary overlap/duplication, as well as overcommitment with the project being proposed. Note that there is no page limitation for this section of the proposal, though some fields have character limitations for consistency and equity.

(b) Senior personnel include the individuals designated by the proposer/awardee organization and approved by NSF who contribute in a substantive, meaningful way to the scientific development or execution of a research and development project proposed to be carried out with a research and development award.[31]

(c) Current and Pending (Other) Support must be provided separately for each individual designated as senior personnel through use of SciENcv. SciENcv will produce an NSF-compliant PDF version of the Current and Pending (Other) Support. Senior personnel must prepare, save, certify, and submit these documents as part of their proposal via Research.gov or Grants.gov.

(d) Consistent with NSPM-33, senior personnel are required to disclose contracts associated with participation in programs sponsored by foreign governments, instrumentalities, or entities, including foreign government-sponsored talent recruitment programs[32]. Further, if an individual receives direct or indirect support that is funded by a foreign government-sponsored talent recruitment program, even where the support is provided through an intermediary and does not require membership in the foreign government-sponsored talent recruitment program, that support must be disclosed. Senior personnel must also report other foreign government sponsored or affiliated activity. Note that non-disclosure clauses associated with these contracts are not acceptable exemptions from this disclosure requirement.

(e) A table entitled, NSPM-33 Implementation Guidance Pre- and Post-award Disclosures Relating to the Biographical Sketch and Current and Pending (Other) Support[33]has been created to provide helpful reference information regarding pre-award and post-award disclosures. The table includes the types of activities to be reported, where such activities must be reported in the proposal, as well as when updates are required in the proposal and award lifecycle. A final column identifies activities that are not required to be reported.

(f) Do not submit any personal information in the Current and Pending (Other) support. This includes items such as: home address; home telephone, fax, or cell phone numbers; home e-mail address; driver’s license number; marital status; personal hobbies; and the like. Such personal information is not appropriate for current and pending (other) support and is not relevant to the merits of the proposal. NSF is not responsible or in any way liable for the release of such material.

(g) A separate submission must be provided for each active project/pending proposal as well as in-kind contributions using the format specified below.

The formats of Current and Pending (Other) Support are as follows:

*= required

(i) Identifying Information
*Name: Enter the name of the senior person (Last name, First name, Middle name, including any applicable suffix).

ORCID ID[34] (Optional): Enter the ORCID ID of the senior person.

*Position Title: Enter the current position title of the senior person.

(ii) Organization and Location
*Name: Enter the name of the primary organization of the senior person.

*Location: Enter the City, State/Province, and Country where the primary organization is located. If the State/Province is not applicable, enter N/A.

(iii) Projects/Proposals
In this section, disclose ALL existing projects, as well as all projects currently under consideration for funding, in accordance with the definitions for “current” and “pending” below. Unless otherwise specified, there is no page or character limit.

*Project/Proposal Title: Enter the title of each project/proposal being reported.

*Status of Support: Select the appropriate status type as defined below:

Current – all ongoing projects, or projects with ongoing obligations, from whatever source irrespective of whether such support is provided through the proposing organization or is provided directly to the individual.
Pending – any proposal currently under consideration for funding (including this proposal) from whatever source irrespective of whether such support is provided through the proposing organization or is provided directly to the individual.
Proposal/Award Number (if available): Enter the applicable proposal/award number for each proposal and/or award, if available.

*Source of Support: Identify the entity for each proposal and/or award that is providing the support. Include all Federal, State, Tribal, territorial, local, foreign, public, or private foundations, non-profit organizations, industrial or other commercial organizations, or internal funds allocated toward specific projects.

*Primary Place of Performance: Identify the primary location where the project or activity is being executed. Enter the City, State/Province, and Country where the organization is located. If the State/Province is not applicable, enter N/A. Indicate “virtual” if the project is not based in a physical location.

*Project/Proposal Start Date: Indicate the start date (MM/YYYY) of the project/activity as proposed/approved.

*Project/Proposal End Date: Indicate the end date (MM/YYYY) of the project/activity as proposed/approved.

*Total Award Amount: Enter the total award amount for the entire period of performance, including indirect costs, rounded to the nearest dollar. If the support is in a foreign country’s currency, convert to U.S. dollars at time of submission.

*Person-Month(s) (or Partial Person-Months) Per Year Committed to the Project: Enter how much time the individual anticipates is necessary to complete the scope of work on the proposed project or award. Enter the number of person-months (even if unsalaried) for the current budget period and enter the proposed person-months for each subsequent budget period. If the individual is reporting person-months that span two calendar years, the individual should enter the latter year. For example, if the entry covers the organization’s fiscal year of June 2023 through May 2024, the individual would enter “2024” for the year and include the corresponding person-months as defined and used by the organization in proposals submitted to NSF. If the time commitment is not readily ascertainable, reasonable estimates should be provided.

*Overall Objectives: Provide a brief statement of the overall objectives of the proposal/award. This field is limited to 1500 characters.

*Statement of Potential Overlap: Provide a description of the potential overlap with any pending proposal or award and this proposal in terms of scope, budget, or person-months planned or committed to the project by the individual. If there is no potential overlap, enter N/A in this field.

(iv) In-Kind Contributions
In this section, disclose ALL[35] in-kind contributions related to current and pending support. In-kind contributions include, but are not limited to, office/laboratory space, equipment, supplies, and employee or student resources.

*Status of Support: Select the appropriate status type as defined below:

Current – all in-kind contributions obligated from whatever source irrespective of whether such support is provided through the proposing organization or is provided directly to the individual.
Pending – all in-kind contributions currently under consideration from whatever source irrespective of whether such support is provided through the proposing organization or is provided directly to the individual.
*In-Kind Contribution Start Date: Indicate the start date (MM/YYYY) of the in-kind contribution as proposed/approved.

*In-Kind Contribution End Date: Indicate the end date (MM/YYYY) of the in-kind contribution as proposed/approved.

*Source of Support: Identify the entity(ies) that is/are providing the in-kind contribution. Include, for example, Federal, State, Tribal, territorial, local, foreign, public, or private foundations, non-profit organizations, industrial or other commercial organizations, or internal funds allocated toward specific projects.

*Summary of In-Kind Contribution(s): Provide a summary of the in-kind contribution(s) not intended for use on the project/proposal being proposed to NSF, whether or not it has an associated time commitment. If the time commitment or dollar value is not readily ascertainable, reasonable estimates should be provided.

*Person-Month(s) (or Partial Person-Months) Per Year Associated with the In-kind Contribution: Enter how much time the individual anticipates is necessary to complete the scope of work associated with the in-kind contribution. Enter the number of person-months (even if unsalaried) for the current budget period and enter the proposed person-months for each subsequent budget period. If the individual is reporting person-months that span two calendar years, the individual should enter the latter year. For example, if the entry covers the organization’s fiscal year of June 2023 through May 2024, the individual would enter “2024” for the year and include the corresponding person-months as defined and used by the organization in proposals submitted to NSF. If the time commitment is not readily ascertainable, reasonable estimates should be provided.

*U.S. Dollar Value of In-Kind Contribution: Enter the U.S. dollar value of the in-kind contribution. If the dollar value is not readily ascertainable, reasonable estimates should be provided. If the support is in a foreign country’s currency, convert to U.S. dollars at time of submission, rounded to the nearest dollar.

*Overall Objectives: Provide a brief statement of the overall objectives of the in-kind contribution(s). This field is limited to 1500 characters.

*Statement of Potential Overlap: Provide a description of the potential overlap with any current or pending in-kind contribution and this proposal in terms of scope, budget, or person-months planned or committed to the project by the individual. If there is no overlap, then enter N/A in the field.

(v) *Certification
When the individual signs the certification on behalf of themselves, they are certifying that the information is current, accurate, and complete. This includes, but is not limited to, information related to current, pending, and other support (both foreign and domestic) as defined in 42 U.S.C. §§6605. Misrepresentations and/or omissions may be subject to prosecution and liability pursuant to, but not limited to, 18 U.S.C. §§ 287, 1001, 1031 and 31 U.S.C. §§3729-3733 and 3802.

(h) The individual also must report any proposal, other than the proposal currently being submitted, that will likely be submitted imminently or in the near future.

(i) Prior to making a funding recommendation, the cognizant NSF program officer will request that an updated version of Current and Pending (Other) Support be submitted via Research.gov.
}
