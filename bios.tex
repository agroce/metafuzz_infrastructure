\section{Biographical Sketches}

\cut{
	h. Senior Personnel Documents
(i) Biographical Sketch(es)
Note: The mandate to use SciENcv only for preparation of the biographical sketch will go into effect for new proposals submitted or due on or after October 23, 2023. In the interim, proposers may continue to prepare and submit this document via use of SciENcv or the NSF fillable PDF. NSF, however, encourages the community to use SciENcv prior to the October 2023 implementation.

(a) Senior Personnel
This section of the proposal is used to assess how well qualified the individual, team, or organization is to conduct the proposed activities. A Biographical Sketch (limited to three pages) must be provided separately for each individual designated as senior personnel through use of SciENcv (Science Experts Network Curriculum Vitae). SciENcv will produce an NSF-compliant PDF version of the Biographical Sketch. Senior personnel must prepare, save, certify, and submit these documents as part of their proposal via Research.gov or Grants.gov.

Senior personnel include the individuals designated by the proposer/awardee organization and approved by NSF who contribute in a substantive, meaningful way to the scientific development or execution of a research and development project proposed to be carried out with a research and development award.[27]

A table entitled, NSPM-33 Implementation Guidance Pre- and Post-award Disclosures Relating to the Biographical Sketch and Current and Pending (Other) Support [28] has been created to provide helpful reference information regarding pre-award and post-award disclosures. The table includes the types of activities to be reported, where such activities must be reported in the proposal, as well as when updates are required in the proposal and award lifecycle. A final column identifies activities that are not required to be reported.

Inclusion of additional information beyond that specified below may result in the proposal being returned without review. Do not submit any personal information in the Biographical Sketch. This includes items such as: home address; home telephone, fax, or cell phone numbers; home e-mail address; driver’s license number; marital status; personal hobbies; and the like. Such personal information is not appropriate for the Biographical Sketch and is not relevant to the merits of the proposal. NSF is not responsible or in any way liable for the release of such material. (See also Chapter III.H).

The format of the Biographical Sketch is as follows:

*= required

(1) Identifying Information
(i) *Name: Enter the name of the senior person (Last name, First name, Middle name, including any applicable suffix).

(ii) ORCID ID[29] (Optional): Enter the ORCID ID of the senior person.

(iii) *Position Title: Enter the current position title of the senior person.

(2) *Organization and Location:
(i) Name: Enter the name of the primary organization of the senior person.

(ii) Location: Enter the City, State/Province, and Country where the primary organization is located. If the State/Province is not applicable, enter N/A. Indicate “virtual” if the project is not based in a physical location.

(3) *Professional Preparation
Provide a list of the senior person’s professional preparation (e.g., education and training), listed in reverse chronological order by start date. Include all postdoctoral and fellowship training, as applicable, listing each separately. Also include the baccalaureate degree or other initial professional education.

For each entry provide:

the name of the organization;
the location of the organization: Enter the City, State/Province, and Country where the primary organization is located. If the State/Province is not applicable, enter N/A.
the degree received (if applicable);
the month and year the degree was received (or expected receipt date). For fellowship applicants only, also include the start date of the fellowship; and
the field of study.


(4) *Appointments and Positions
Provide a list, in reverse chronological order by start date, of all the senior person’s academic, professional, or institutional appointments and positions, beginning with the current appointment (including the associated organization and location). Appointments and positions include any titled academic, professional, or institutional position whether or not remuneration is received, and whether full-time, part-time, or voluntary (including adjunct, visiting, or honorary).

For each entry provide:

Start date: YYYY
End date: YYYY
Appointment or Position Title
Name of organization
Department (if applicable)
Location of organization: City, State/Province, Country
With regard to professional appointments, senior personnel must only identify all current domestic and foreign professional appointments outside of the individual's academic, professional, or institutional appointments at the proposing organization.

(5) *Products
Provide a list of: (i) up to five products most closely related to the proposed project; and (ii) up to five other significant products, whether or not related to the proposed project that demonstrate the senior person’s qualifications to carry out the project as proposed. Acceptable products must be citable and accessible, including but not limited to:

publications, conference papers, and presentations;
website(s) or other Internet site(s);
technologies or techniques;
inventions, patent applications, and/or licenses; and
other products, such as data, databases, or datasets, physical collections, audio or video products, software, models, educational aids or curricula, instruments or equipment, research material, interventions (e.g., clinical or educational), or new business creation.
Only the list of ten will be used in the review of the proposal.

Each product must include full citation information including:

names of authors;
product title;
date of publication or release;
website URL;
other persistent identifier (if available); and
other relevant citation information (e.g., in the case of publications, title of enclosing work such as journal or book, volume, issue, pages).
If any of the items specified above is not applicable, enter N/A.

Senior personnel who wish to include publications in the products section of the Biographical Sketch that include multiple authors may, at their discretion, choose to list one or more of the authors and then "et al" in lieu of including the complete listing of authors' names.

(6) *Synergistic Activities
Provide a list of up to five distinct examples that demonstrates the broader impact of the individual’s professional and scholarly activities that focus on the integration and transfer of knowledge as well as its creation. Examples may include, among others: innovations in teaching and training; contributions to the science of learning; development and/or refinement of research tools; computation methodologies and algorithms for problem-solving; development of databases to support research and education; broadening the participation of groups underrepresented in STEM; participation in international research collaborations; participation in international standards development efforts; and service to the scientific and engineering community outside of the individual’s immediate organization.

Synergistic activities must be specific and must not include multiple examples to further describe the activity. Examples with multiple components, such as committee member lists, sub-bulleted highlights of honors and prizes, or a listing of organizations for which the individual has served as a reviewer, are not permitted.

(7) *Certification
When the individual signs the certification on behalf of themselves, they are certifying that the information is current, accurate, and complete. This includes, but is not limited to, information related to domestic and foreign appointments and positions. Misrepresentations and/or omissions may be subject to prosecution and liability pursuant to, but not limited to, 18 U.S.C. §§287, 1001, 1031 and 31 U.S.C. §§3729-3733 and 3802.

(b) Other Personnel
For the personnel categories listed below, the proposal also may include information on exceptional qualifications that merit consideration in the evaluation of the proposal. While the requirement to use SciENcv for preparation and submission of the biographical sketch is for any individual designated as senior personnel, the biographical information for 'other personnel' may be freeform. The biographical information must be clearly identified as “Other Personnel” biographical information and uploaded as a single PDF file in the Other Supplementary Documents section of the proposal.

(1) Postdoctoral associates

(2) Other professionals

(3) Students (research assistants)

(c) Equipment Proposals
For equipment proposals, the following must be provided for each auxiliary user:

(1) Short biographical sketch; and

(2) List of up to five publications most closely related to the proposed acquisition.

Such information should be clearly identified as “Equipment Proposal” biographical information and uploaded as a single PDF file in the Other Supplementary Documents section of the proposal.
	}
