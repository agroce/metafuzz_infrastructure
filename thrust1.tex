\cut{Thrust 1: A Foundation for Efficient Fuzzer Execution,
Orchestration, and Evaluation}

An infrastructure for (meta-)fuzzing research, 
whether for evaluating large sets of fuzzers in a (meta-)fuzzing approaches, 
evaluating architecture-specific fuzzing feature benefits, 
or for less-explored but promising research such as determining which kinds of 
fuzzers work best for broad classes of fuzzing, needs to be sufficiently 
efficient, robust and configurable to support a variety of (possibly 
unforeseen) requirements.  
Further, this infrastructure needs to be easily extendable to support new input 
mutators, benchmarks, input sharing, and new computation-providing backends 
(e.g., a new version of OpenStack.)

We propose an infrastructure, as shown in Figure~\ref{fig:overview} that meets these many various needs by focusing on 
simplicity and extensibility.  We propose a simple API based on separating the 
provisioning of hardware resources that installs an execution agent on the 
provisioned hardware.   The execution agent, running with elevated privileges, 
talks to a centralized server to coordinate a fuzzing.  It reports back 
available hardware resources.  The central server instructs execution agents on 
packages to install, fuzzing campaigns to start, fuzzing parameters such as 
which mutator to use,, during- and post-campaign statics to request, etc.  The 
fuzzing-campaign director interacts with the server to configure, start, 
monitor, interrupt, or gather statistics from a fuzzing campaign.  Further, a 
campaign director can upload a previously created configuration to the server 
to launch a campaign, allowing for repeatable operation..  

The infrastructure will come with pre-configured fuzzing campaigns, fuzzer 
configurations, input sharing schemas, and VM images to allow easy adoption for 
beginning users.  We plan to initially target OpenStack leveraging the UVA 
hardware resources available, as well as CloudBank AWS/Azure.

Lastly, the configuration options will focus on general meta-fuzzing strategies 
to allow for easy user extensibility.  End-users will be able to easily create 
new fuzzing components (e.g., input mutator, code instrumentation type, input 
sharing mechanisms) by writing new configuration files, schemas or 
pre-configured machine images.  The collaborative effort will attempt to accept 
community-submitted extensions to our proposed open-source infrastructure when 
the project has achieved a minimal level of acceptance.

The following thrusts describe in more detail how meta-fuzzing can be realized 
in the proposed infrastructure.


\begin{figure}[htbp!]
%\makebox[7in][7in]{to be filled in}
\vbox to 7in {\vfil
\hbox to 7in{a figure to be drawn later}%
\vfil
}

\caption{Proposed Meta-fuzzing infrastucture overview.  
A fuzz-campaign director interacts with the meta-fuzzer server to setup or replay a configured fuzzing campaign.  
Built-in collections of of benchmarks, machine images, input sharing techniques, hardware resource requirements, input mutators, and other fuzzer configuration 
allow for quick startup of fuzzing campaigns backended by popular cloud service providers as well as configurability and extensibility for custom fuzzing setups.  
The system allows an experimentor to quickly and simply obtain results, or reproduce prior fuzzing campaigns.}
\label{fig:overview}
\end{figure}



\cut{

	\subsection{Thrust 1: A Foundation for Efficient Fuzzer Execution,
Orchestration, and Evaluation}

The execution of large sets of fuzzers for purposes of evaluating
(meta-)fuzzing approaches, or for less-explored but promising research such as
determining which kinds of fuzzers work best for broad classes of fuzzing
targets (e.g., compilers, media parsers, data structures, etc.) is essentially
the same core task as that required for sophisticated ensemble fuzzing.  The
only real difference is that a benchmarking run 1) runs all the fuzzers with
the same resource allocation and 2) there is no sharing of inputs between
fuzzers.  That is, fuzzer benchmarking is a simplified, restricted, version of
what is needed for a successful platform for implementing ensemble fuzzing.
The core needs are the same: ease of adding new fuzzers and including new
targets, ease of deployment to both local environments (for easy debugging or
use in developer ``unit fuzzing'') and the cloud (for large-scale fuzzing or
fuzzer evaluation), and ability to produce easily interpretable, statistically
usable, results (for both human consumption and resource allocation).

\jdh{UVA can you talk more about how to go about this thrust?}

fuzzers -- which fuzers to use, and how to extend infrastructure for new 
fuzzers.  This is really NAU?
benchmarks -- standard fuzzing benchmarks, plus ability to add custom 
benchmarks.
specification -- which backend to use, how to allocate resources, how a fuzzing 
campaign should look like, which mutators to use, which instrumentation to use.
deployment -- VM allocation, software installation, efficient (semi-)automatic 
assignment of fuzzers based on fuzz campaign mapping
cooperation -- how often to share, and what to share.  Sharing "architectures" 
(e.g., share on the same VM, vs. different VMs in same cluster vs VMs in 
different clusters vs VMs in different data centers)
results -- collecting results (baseline results + user-specified extensions), 
statistical analysis


}
