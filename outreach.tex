\documentclass[12pt]{article}

\usepackage{fullpage}

\pagenumbering{gobble}

\begin{document}

\begin{center} {\Large\sf\textbf{Collaborative Research: CIRC: New: Medium: A
      Development and Experimental Environment for Meta-Fuzzing}}
\end{center}

\section*{Community Outreach Documentation}


\begin{table}[b]
  \begin{tabular}{|l|p{10cm}|l|}
    {\bf Year} & {\bf Outreach and Participation} & {Budget} \\
    \hline
    1 & Contact key experts in fuzzer evaluation and early meta-fuzzing-like
        approaches not involved in this project & N/A \\
    2 & Continue interaction with outside experts, invite other teams to try
        prototype infrastructure & N/A \\
    3 & Continue interactions, also hold a small workshop for key interested
        experts in academia and industry, at NAU, early in year 3 to ensure
        valdiity of final infrastructure decisions & \$17,000 \\
  \end{tabular}
  \caption{Community Outreach and Participation by Project Year}
  \label{tab:outreach}
  
\end{table}


Table \ref{tab:outreach} shows the overall plan for community outreach.  During
the first and second years, outreach and collaboration will primarily be in the
form of contacting key experts in fuzzer evaluation, and researchers with
interests in meta-fuzzing-like approaches to ensure their input on the
community's needs are taken into account.  We plan to share design and
implementation decisions that might effect the community as early as possible,
in order to make the infrastructure responsive to user needs.  The PIs have
contact with a large number of key contributors to fuzzing research in industry
and academia, and have begun discussions of these problems with some of those
contacts.

At the beginning of the third year, we plan to hold a workshop at Northern
Arizona University and invite those who have shown interest in the
infrastructure and provided critical advice, for an in-depth hands-on dive into
working prototypes, and finalize decisions.  This workshop can also serve to
provide publicity for the availability of the infrastructure, and solidify the
advisory council's relationship to the project, as well as lay grounds for
sustainable maintenance of the infrastructure.


\end{document}
