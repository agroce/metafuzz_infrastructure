%\clg{from the call: ``describe services to be integrated into the infrastructure, including 
%mechanisms by which researchers will gain access to the infrastructure;''}
%\clg{I'm wondering if one of the subsections I initially grouped under 
%tools/resources/datasets might fit better down here.}

% User Services

A good first impression will be key in engendering trust in our meta-fuzzing infrastructure.
Thus, we will provide ready-made images (Docker or VM) so that users can experience
the end-to-end system depicted in Figure~\ref{fig:overview}.
We will provide a suite of such images, pre-configured with various combination of
fuzzers (e.g., AFL++, LibAFL, libFuzzer), 
fuzzing styles (e.g., whole program, library-based, source, binary), 
and benchmarks (e.g, OSS Fuzz benchmarks, programs written in different languages).

We will actively develop and maintain user documentation, cookbooks and tutorials throughout
the duration of the project. We will support both users that want to just use the system
as well as those researchers that wish to extend the system.
In general, our tutorials and documentation of meta-fuzzing workflows will be
represented and concretized in ready-made images. 

We will host our software repository on github.com and use standard GitHub facilities
to manage interactions with end users, e.g., issues, pull requests, discussions, feature requests.
Over time, other researchers and practitioners will extend the base infrastructure with
new fuzzers, benchmarks, transformations, heuristics to guide meta-fuzzing, etc. 
While we will initially serve as gatekeepers to decide when contributions from the community
should go into the mainline release, we will coordinate with our advisory board and active community 
members to articulate governance policies for the project.

Note that unlike OSS Fuzz, our primary aim is not to provide fuzzing-as-a-service, but
rather, to provide the software necessary to enable easy-to-deploy and extend meta-fuzzing 
capabilities (though, of course, our software stack could be used to power such a service).

