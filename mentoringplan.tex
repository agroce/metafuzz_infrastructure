\documentclass[12pt]{article}


\usepackage{fullpage}

\begin{document}

\pagenumbering{gobble}

\begin{center}
{\Large\sf\textbf{Collaborative Research: CIRC: New: Medium: A Development and 
Experimental Environment for Meta-Fuzzing}}
\end{center}

\paragraph{Overview:} PIs Groce, Le Goues, Padhye, [YOU FOLKS] are experienced mentors of graduate students and junior researchers. PI Groce has been the chair of the doctoral committee for five students who have received doctorates in computer science, and has served or is serving on the committee for nine other PhD students, as well as mentoring 16 graduated masters' students. PI Groce also supervised a post-doctoral researcher as part of a collaborative NSF proposal with the University of Utah.  PI Le Goues has advised graduated 7 PhD students and presently advises 9 matriculated PhD students; she has also supervised 3 post-doctoral researchers; all at Carnegie Mellon.  She has also served on 13 PhD committees for students she does not directly advise, at CMU and elsewhere. PI Padhye is currently advising 3 PhD students and has served on 3 PhD thesis committees (two at CMU and one external) as well as 1 Master's student committee.


The PIs plan to apply methods they have previously applied to mentoring students. All mentees are expected to participate heavily in the conception, development, experimentation, and communication aspects of infrastructure tasks, under the guidance of the PIs. The mentoring plan is grounded in the guidance provided by the National Academies of Science
and Engineering, and on widely-accepted best practices for mentoring graduate research students.


\paragraph{Travel:}  Graduate research students will all attend at least one event to  publicize the project, and ideally will each attend two events.  Outreach activities of students will bring them in contact with leading figures in the fields of software engineering and computer security.

\paragraph{Networking:}  More generally, the nature of this project with large outreach and community communication (including the Fuzzing Community Slack) will introduce graduate students to leading researchers in academia and industry.

\paragraph{Teaching:}  The methods of fuzzer evaluation are already a topic of advanced (and undergraduate) classes taught by PI Groce, and the concept of ensemble fuzzing can be demonstrated in class by graduate students on the project.  PIs Le Goues and Padhye both teach courses on Program Analysis, including a course on Domain-specific analyses; both offer opportunities for graduate students to participate or TA. 

\paragraph{Grant Proposals:}  Graduate research students will be involved in writing follow-on proposals, if applicable.

\paragraph{Building a strong publication record:}  The PIs will advise the student in planning publications in workshops, conferences, and journals, according to their progress in the program. This will help them develop their research programs and make them known among peers in the field. Through writing papers with the PI, the student is expected to improve their critical thinking, writing, and scientific skills.


\end{document}