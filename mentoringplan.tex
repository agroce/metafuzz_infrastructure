\documentclass[12pt]{article}


\usepackage{fullpage}

\begin{document}

\pagenumbering{gobble}

\begin{center}
{\Large\sf\textbf{Collaborative Research: CIRC: New: Medium: A Development and 
Experimental Environment for Meta-Fuzzing}}
\end{center}

\paragraph{Overview:} PIs Groce, Gerosa, and Steinmacher are experienced mentors of graduate students and junior researchers. PI Groce has been the chair of the doctoral committee for five students who have received doctorates in computer science, and has served or is serving on the committee for eight other PhD students, as well as mentoring 16 graduated masters' students. PI Groce also supervised a post-doctoral researcher as part of a collaborative NSF proposal with the University of Utah.  Dr. Gerosa graduated 10 PhDs and co-supervised other 2. Dr. Steinmacher graduated 2 PhDs as co-supervised 5. Dr. Steinmacher was the main supervisor of the recipient of the prestigious ACM SIGSOFT Distinguished Dissertation Award 2024 (received by Bianca Trinkenreich, co-supervised by Dr. Gerosa).


The PIs plan to apply methods they have previously applied to mentoring students. All mentees are expected to participate heavily in the conception, development, experimentation, and communication of research results, under the guidance of the PIs. The mentoring plan is grounded in the guidance provided by the National Academies of Science
and Engineering, and on widely-accepted best practices for mentoring graduate research students.

\paragraph{Travel:}  Graduate research students will all attend at least one event to present research results or publicize the project, and ideally will each attend two events.  

\paragraph{Teaching:}  The DeepState tool is already incorporated into PI Groce's teaching on software testing and analysis, and graduate students will help engage students with the tool, to encourage some students to become contributors to the project.

\paragraph{Grant Proposals:}  Graduate research students and postdoctoral researchers will be involved in writing follow-on research proposals, if applicable.

\paragraph{Building a strong publication record:}  The PIs will advise the student in planning publications in workshops, conferences, and journals, according to their progress in the program. This will help them develop their research programs and make them known among peers in the field. Through writing papers with the PI, the student is expected to improve their critical thinking, writing, and scientific skills.


\end{document}