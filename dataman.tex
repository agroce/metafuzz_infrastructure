\documentclass[12pt]{article}

\usepackage{fullpage}

\pagenumbering{gobble}

\begin{document}

\begin{center}
{\Large\sf\textbf{Collaborative Research: CIRC: New: Medium: A Development and 
    Experimental Environment for Meta-Fuzzing
  }}
\end{center}

\begin{center}
{\sf\textbf{Data Management Plan
  }}
\end{center}
The data in the proposed project is primarily source code for
infrastructure, plus documentation, tests, and benchmark
examples. These artifacts  will be
released in a project GitHub repository (or more likely, multiple
respositories under one project ``ID'') to be created. Any experimental data should be reproducible using the code artifacts, but will also be published in a
condensed form in the repository, and/or other standard data archival
sites used by the scientific community (e.g., some conferences have
particular preferences for artifact data hosting).

Curricular materials and documentation associated with any tools will also be stored in
an open source repository, since in this project the primary educational benefits are linked
to the use of these tools.

Using GitHub automatically provides us with excellent backup and
archiving for the code and curricular material products of the project.
We have no unusual format or metadata requirements; the tools involved primarily use
textual formats that are easy to store and read or byte-based formats that are interpreted
only by fuzzer drivers or code under test.

We expect to release all software under standard open source licenses
allowing for a wide range of uses in academic and industrial settings,
e.g., the MIT License.

Some of the expertise and rationale for design decisions will be
developed in the context of workshops or Slack discussions; we will
archive all critical documents and inputs in longer-term formats in
documentation for software releases, since in this project, design
decisions and examples of connections to existing fuzzers are an
important, if secondary, product.

\end{document}
