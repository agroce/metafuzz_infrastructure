
\section{Cover Sheet}

\noindent \bold{Start Date \& Duration}: 
\\

\noindent \bold{Letter of Intent}: N/A
\\

\noindent \bold{Related Preliminary Proposal}: N/A
\\

\noindent \bold{Prime Organization}: Northern Arizona University
\\

\noindent \bold{Awardee Organization}:    \\
\hspace{1in}Name \\
\hspace{1in}Address \\
\hspace{1in}NSF Organization Code: XYZ \\
\hspace{1in}UEI \\
\hspace{1in}Employer Identification Number/Taxpayer Identification Number:  \\
\hspace{1in}For-profit Organization: No\\
\hspace{1in}Small Business: No\\
\hspace{1in}Minority Business: No\\
\hspace{1in}Woman-owned Business: No\\
\\

\noindent \bold{Primary Place of Performance Organization}:    \\
\hspace{1in}University of Virginia \\
\hspace{1in}United States \\
\hspace{1in}Street Address\\
\hspace{1in}Charlottesville, VA 22904-XXXX\\
\\

\noindent \bold{Other Federal Agencies}: None
\\

\noindent \bold{Other Information}: None
/\

\cut{
There are seven components of the Cover Sheet. The Cover Sheet data elements are as follows:

Requested Start Date and Proposal Duration
The proposed duration for which support is requested should be consistent with the nature and complexity of the proposed activity. The Foundation encourages proposers to request funding for durations of three to five years when such durations are necessary for completion of the proposed work and are technically and managerially advantageous. The requested start date should allow at least six months for NSF review, processing, and decision. PIs should consult their organization’s SPO for unusual situations (e.g., a long lead time for procurement) that create problems regarding the proposed start date. Specification of a desired start date for the project is important and helpful to NSF staff; however, requests for specific start dates may not be met.

Related Letter of Intent (LOI)
If an LOI was submitted, enter the LOI ID number that was issued upon submission.

Related Preliminary Proposal
If a preliminary proposal was submitted, and the organization was either invited or encouraged/discouraged to submit a full proposal, provide the Preliminary Proposal Number.

Prime Organization
The information on the Awardee Organization is prefilled on the Cover Sheet based on the login information entered. NSF uses the legal business name and physical address from the organizations’ SAM registration.

The awardee organization name, address, NSF organization code, UEI, and Employer Identification Number/Taxpayer Identification Number are derived from the profile information provided by the organization or pulled by NSF from the SAM database and are not entered when preparing the Cover Sheet.

Organizations must identify their status by checking all the applicable boxes on the Cover Sheet:

For-profit organizations must be U.S.-based commercial organizations, including small businesses, with strong capabilities in scientific or engineering research or education and a passion for innovation. See PAPPG Chapter I.E.3 for additional information.

A small business must be organized for profit, independently owned, and operated (not a subsidiary of, or controlled by, another firm), have no more than 500 employees, and not be dominant in its field.

A minority business must be: (i) at least 51 percent owned by one or more minority or disadvantaged individuals or, in the case of a publicly owned business, have at least 51 percent of the voting stock owned by one or more minority or disadvantaged individuals; and (ii) one whose management and daily business operations are controlled by one or more such individuals.

A woman-owned business must be at least 51 percent owned by a woman or women, who also control and operate it. "Control" in this context means exercising the power to make policy decisions. "Operate" in this context means being actively involved in the day-to-day management.

Primary Place of Performance
The Primary Place of Performance (PPoP) information will default to the organization’s physical address. If the project will be performed at a location other than the awardee organization, provide the following information (where applicable).

Organization Name (identify the organization name of the primary site where the work will be performed, if different than the awardee);
Country
Street Address;
City;
State/Territory; and
9-digit Postal Code.
Note that not all fields listed above are required. Research.gov specifies the fields that are required for projects that will be performed at locations other than that of the proposing organization.

For research infrastructure projects, the project/performance site should correspond to the physical location of the asset. For research infrastructure that is mobile or geographically distributed, information for the primary site or organizational headquarters (as defined by the proposer) should be provided.

Other Federal Agencies
If the proposal is being submitted for consideration by another Federal agency, the abbreviated name(s) of the Federal agency(ies) must be identified in the space provided.

Other Information
If any of the following items on the Cover Sheet are applicable to the proposal being submitted, the relevant box(es) must be checked.

Beginning Investigator (See Chapter II.E.2) (Note: this box is applicable only to proposals submitted to the Biological Sciences Directorate.)

Disclosure of Lobbying Activities (See Chapter II.D.1.d)

Proprietary or Privileged Information (See Chapter II.D.1.c and II.E.1)

Special Exceptions to the Deadline Date Policy (See Chapter I.F.3)

Historic Places (See Chapter II.D.2.i(vii))

Live Vertebrate Animals[12] (See Chapter II.E.4)

Human Subjects[13] (See Chapter II.E.5)

Funding of an International Branch Campus of a U.S. IHE (See Chapter I.E.1) – If this box is checked, the proposer also must enter the name of the applicable country(ies) in the International Activities Country Name(s) box described below.

Funding of a Foreign Organization or Foreign Individual (See Chapter I.E.6) – If this box is checked, the proposer also must enter the name of the applicable country(ies) in the International Activities Country Name(s) box described below.

International Activities Country Name(s) – each proposal that describes an international activity, proposers should list the primary countries involved. An international activity is defined as research, training, and/or education carried out in cooperation with international counterparts either overseas or in the U.S. using virtual technologies. Proposers also should enter the country/countries with which project participants will engage and/or travel to attend international conferences. If the specific location of the international conference is not known at the time of the proposal submission, proposers should enter “Worldwide”. (See Chapter II.E.8)

Potential Life Sciences Dual Use Research of Concern (DURC) (See Chapters II.E.6 and XI.B.5)

Off-Campus or Off-Site Research – For purposes of this requirement, off-campus or off-site research is defined as data/information/samples being collected off-campus or off-site, such as fieldwork and research activities on vessels and aircraft. (See Chapter II.D.1.d(viii) and II.E.9).
}
