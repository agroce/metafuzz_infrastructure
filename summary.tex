\centerline{\noindent{\large{\textbf{A. Project Summary}}}}


\cut{
}

\subsection*{Overview}
\vspace{-2mm}
The complexity of modern software systems, and the need for widely used 
libraries and other ``code infrastructure'' to be reliable, demands effective 
\emph{testing}.  Bugs in code have increasing impact on society at large (e.g., 
numerous security breaches traceable to incorrect code have very high economic 
cost).  Testing remains to this day the single most effective means of finding 
bugs when their incurred costs are low.   Moreover, fuzzing, originally limited largely to 
the software security world, is the most promising new approach to massively 
automated and effective testing of software.   However, to date most fuzzing 
research, even if theoretically able to generalize to multiple underlying 
fuzzers, has been focused on \emph{producing a new fuzzer}.  That is, if a 
researcher (or security developer at, e.g., Google) devises a way to improve 
the effectiveness of fuzzing, the common practice is to implement a new, 
competing fuzzer, or modify a well-know existing fuzzer.  In some cases, this 
is appropriate: the approach defines a fuzzing technique.  However, the most 
promising fuzzing advances likely are \emph{meta-fuzzing} approaches that 
improve the performance, potentially, of \emph{all} fuzzers, without requiring 
modification of the underlying fuzzer itself.  Such techniques primarily 
include those that modify the source or binary of the fuzzing target to help 
the fuzzer, and techniques that make use of multiple fuzzers as ``black boxes'' 
that contribute to an ensemble fuzzing effort.  This proposal addresses the 
serious under-examination of such techniques, by aiming to construct a 
framework for development of meta-fuzzing approaches of these categories, 
making it easy to apply a technique to multiple fuzzers, and making it easy to 
evaluate/benchmark such techniques.  The ability to evaluate methods is 
necessary to enable ensemble methods that weight fuzzers by performance, and so 
is required for a meta-fuzzing development platform, in any case.

\subsection*{Intellectual Merit} 
\vspace{-2mm}
The importance of fuzzing to future efforts to find subtle bugs in complex 
software systems is well-known.  This proposal aims primarily to boost research 
and development in the area of fuzzing methods that can work \emph{for any 
fuzzer}, including future, better, underlying fuzzing algorithms.  As part of 
that work, it also aims to standardize and advance the evaluation of fuzzing 
algorithms, at present a major bottleneck in fuzzing science advances.  We 
further expect meta-fuzzing approaches to be particular effective in improving 
understanding of general program dynamics, rather than 
tool-engineering-oriented fuzzing techniques.
\subsection*{Broader Impacts}
\vspace{-2mm}

Correct software is increasingly important in our modern digital/networked 
society since ``software is eating the world'' and this process shows no sign 
of stopping or even decelerating. 
Our project will benefit every area of society where software is employed since 
our project aims to improve a fundamental method for effective software testing.
We expect that software security in particular will be advanced by the 
improvements made to the performance of fuzzing methods, and that meta-fuzzing 
may also help support the emerging focus on applying fuzzing to more functional 
properties of code.

\paragraph{Keywords:}
CISE; fuzzing, meta-fuzzing, ensemble methods, program transformations
