\centerline{\noindent{\large{\textbf{A. Project Summary}}}}


\cut{
}

\subsection*{Overview}
\vspace{-2mm}
The complexity of modern software systems, and the need for widely used 
libraries and other ``code infrastructure'' to be reliable, demands effective 
\emph{testing}.  Testing remains to this day the single most effective means of finding 
bugs when their incurred costs are low.   Moreover, fuzzing, originally limited largely to 
the software security world, is the most promising new approach to massively 
automated and effective testing of software.   However, to date most fuzzing 
research, even if theoretically able to generalize to multiple underlying 
fuzzers, has been focused on \emph{producing a new fuzzer}.  If a 
researcher or security practitioner devises a way to improve 
the effectiveness of fuzzing, the common practice is to implement a new, 
competing fuzzer, or modify a well-known existing fuzzer.  In some cases, this 
is neccessary.  However, the most 
promising fuzzing advances are often \emph{meta-fuzzing} approaches that 
improve the performance of \emph{all} fuzzers, without requiring 
modification of the underlying fuzzer.  Such techniques primarily 
include those that modify the source or binary of the fuzzing target to help 
the fuzzer, and techniques that make use of multiple fuzzers as ``black boxes'' 
that contribute to an ensemble fuzzing effort.  This proposal addresses the 
serious under-examination of these techniques, by aiming to construct a 
framework for development and \emph{evaluation} of meta-fuzzing approaches of
these categories.   The ability to evaluate methods is 
necessary to enable ensemble methods that weight fuzzers by
performance, and more generally benefits the entire fuzzing research community.

\subsection*{Intellectual Merit} 
\vspace{-2mm}
The importance of fuzzing to future efforts to find subtle bugs in complex 
software systems is well-known.  This proposal aims primarily to boost research 
and development in the area of fuzzing methods that can work \emph{for any 
fuzzer}, including future, better, underlying fuzzing algorithms.  As part of 
that work, it also aims to standardize and advance the evaluation of fuzzing 
algorithms, at present a major bottleneck in fuzzing science advances.  We 
further expect meta-fuzzing approaches to be particularly effective in improving 
understanding of general program dynamics.

\subsection*{Broader Impacts}
\vspace{-2mm}
Correct software is increasingly important in our modern digital/networked
society and this process shows no sign
of stopping or even decelerating.
This project will benefit every area of society where software is employed because it aims to improve a fundamental method for effective software testing.
Software security, in particular, will be advanced by the
improvements made to  fuzzing methods, and meta-fuzzing
will support an emerging focus on fuzzing for functional
properties of code.
A significant part of this effort is building a diverse and inclusive community of fuzzing and software testing researchers that work together to support the advance of fuzzing science.
Consequently, community outreach, education, and dissemination activities are major components of this effort.
Diversity-enhancing activities for this project are planned through
the diversity, equity, and inclusion centers at CMU, NAU and UVA.  NAU has been designated as a Hispanic serving-institution, with over 25\% student Hispanic enrollment.
Both CMU and UVA have verified broadening participation plans (BPCs).
CMU's BPC activities include developing undergraduate student scaffolding programs, events, and conferences that include a focus on undergraduate research and education experiences.
UVA is fortunate to have five nearby HBCUs: Hampton University, Norfolk State University, Virginia State University, Virginia Union University, and Virginia University of Lynchburg.
As part of UVA's BPC activities, CS faculty have established long-term relationships to encourage student engagement in computer science education and research.
Co-PI Davidson currently has a joint project with Norfolk State that involves undergraduate students in cybersecurity and previously had a joint cybersecurity education project with Virginia State.

\paragraph{Keywords:}
CISE; fuzzing; meta-fuzzing; ensemble methods; program transformations
