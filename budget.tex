
\cut{
	f. Budget and Budget Justification
The proposal budget sets forth how much money the proposer is requesting, by category, to complete the project. The budget justification provides a more detailed breakdown of proposed spending in each category as well as a justification supporting the numbers provided in each budget category. This information is relied upon by NSF in formulating the total award amount and final award budget that is incorporated into any resultant award. (See PAPPG Chapter VI.B.1.)

Each proposal must contain a budget for each year of support requested. The budget justification must be no more than five pages per proposal. The amounts for each budget line item requested must be documented and justified in the budget justification as specified below.

For proposals that contain a subaward(s), each subaward must include a separate budget justification of no more than five pages. See Chapter II.D.2.f(vi)(e) for further instructions on proposals that contain subawards. For collaborative proposals submitted by multiple organizations, each organization must include a separate budget justification of no more than five pages.

The proposal may request funds under any of the categories listed so long as the item and amount are considered necessary, reasonable, allocable, and allowable under 2 CFR §200, Subpart E, NSF policy, and/or the program solicitation. For-profit entities are subject to the cost principles contained in the Federal Acquisition Regulation, Part 31. Amounts and expenses budgeted also must be consistent with the proposing organization's policies and procedures and cost accounting practices used in accumulating and reporting costs.

Proposals for mid-scale and major facilities also should consult NSF’s Research Infrastructure Guide as well as the relevant solicitation for additional budgetary preparation guidelines.

(i) Salaries and Wages (Lines A and B on the Proposal Budget)
(a) Senior Personnel Salaries & Wages Policy
NSF regards research as one of the normal functions of faculty members at institutions of higher education. Compensation for time normally spent on research within the term of appointment is deemed to be included within the faculty member’s regular organizational salary.

As a general policy, NSF limits the salary compensation requested in the proposal budget for senior personnel to no more than two months of their regular salary in any one year. (See Exhibit II-3 for the definitions of Senior Personnel.) It is the organization’s responsibility to define and consistently apply the term “year”, and to specify this definition in the budget justification. This limit includes salary compensation received from all NSF-funded grants. This effort must be documented in accordance with 2 CFR §200, Subpart E, including 2 CFR §200.430(i). If anticipated, any compensation for such personnel in excess of two months must be disclosed in the proposal budget, justified in the budget justification, and must be specifically approved by NSF in the award notice budget.[14]

Under normal rebudgeting authority, as described in Chapters VII and X, a recipient can internally approve an increase or decrease in person months devoted to the project after an award is made, even if doing so results in salary support for senior personnel exceeding the two-month salary policy. No prior approval from NSF is necessary unless the rebudgeting would cause the objectives or scope of the project to change. NSF prior approval is necessary if the objectives or scope of the project change.

These same general principles apply to other types of non-academic organizations.

(b) Administrative and Clerical Salaries & Wages Policy
In accordance with 2 CFR §200.413, the salaries of administrative and clerical staff should normally be treated as indirect costs (F&A). Direct charging of these costs may be appropriate only if all the conditions identified below are met:

(i) Administrative or clerical services are integral to a project or activity;

(ii) Individuals involved can be specifically identified with the project or activity;

(iii) Such costs are explicitly included in the approved budget or have the prior written approval of the cognizant NSF Grants and Agreements Officer; and

(iv) The costs are not also recovered as indirect costs.

Conditions (i) (ii) and (iv) above are particularly relevant for consideration at the budget preparation stage.

(c) Procedures
The names of the PI(s), faculty, and other senior personnel and the estimated number of full-time-equivalent person-months for which NSF funding is requested, and the total amount of salaries requested per year, must be listed. For consistency with the NSF cost sharing policy, if person months will be requested for senior personnel, a corresponding salary amount must be entered on the budget. If salary and person months are not being requested for an individual designated as senior personnel, they should be removed from Section A of the budget. Their name(s) will remain on the Cover Sheet and the individual(s) role on the project should be described in the Facilities, Equipment and Other Resources section of the proposal.

For postdoctoral associates and other professionals, the total number of persons for each position must be listed, with the number of full-time-equivalent person-months and total amount of salaries requested per year. For graduate and undergraduate students, secretarial, clerical, technical, etc., whose time will be charged directly to the project, only the total number of persons and total amount of salaries requested per year in each category is required. Compensation classified as salary payments must be requested in the salaries and wages category. Salaries requested must be consistent with the organization’s regular practices. The budget justification should detail the rates of pay by individual for senior personnel, postdoctoral associates, and other professionals.

(d) Confidential Budgetary Information
The proposing organization may request that salary data on senior personnel not be released to persons outside the Government during the review process. In such cases, the item for senior personnel salaries in the proposal may appear as a single figure and the person-months represented by that amount omitted. If this option is exercised, senior personnel salaries and person-months must be itemized in a separate statement and forwarded to NSF in accordance with the instructions specified in Chapter II.E.1. This statement must include all of the information requested on the proposal budget for each person involved. NSF will not forward the detailed information to reviewers and will hold it privileged to the extent permitted by law. The information on senior personnel salaries will be used as the basis for determining the salary amounts shown in the budget. The box for "Proprietary or Privileged Information" must be checked on the Cover Sheet when the proposal contains confidential budgetary information.[15]

(ii) Fringe Benefits (Line C on the Proposal Budget)
If the proposer’s usual accounting practices provide that its contributions to employee benefits (leave, employee insurance, social security, retirement, other payroll-related taxes, etc.) be treated as direct costs, NSF award funds may be requested to fund fringe benefits as a direct cost. These are typically determined by application of a calculated fringe benefit rate for a particular class of employee (full time or part-time) applied to the salaries and wages requested. They also may be paid based on actual costs for individual employees if that institutional policy has been approved by the cognizant Federal agency. See 2 CFR §200.431 for the definition and allowability of inclusion of fringe benefits on a proposal budget.

(iii) Equipment (Line D on the Proposal Budget)
Equipment is defined as tangible personal property (including information technology systems) having a useful life of more than one year and a per-unit acquisition cost which equals or exceeds the lesser of the capitalization level established by the proposer for financial statement purposes, or $5,000. It is important to note that the acquisition cost of equipment includes modifications, attachments, and accessories necessary to make an item of equipment usable for the purpose for which it will be purchased. Items of needed equipment must be adequately justified, listed individually by description and estimated cost.

Allowable items ordinarily will be limited to research equipment and apparatus not already available for the conduct of the work. General purpose equipment such as office equipment and furnishings, and information technology equipment and systems are typically not eligible for direct cost support. Special purpose or scientific use computers or associated hardware and software, however, may be requested as items of equipment when necessary to accomplish the project objectives and not otherwise reasonably available. Any request to support such items must be clearly disclosed in the proposal budget, justified in the budget justification, and be included in the NSF award budget. See 2 CFR §200.313 and 200.439 for additional information.

(iv) Travel (Line E on the Proposal Budget)
(a) General
When anticipated, travel and its relation to the proposed activities must be specified, itemized, and justified by destination and cost. Funds may be requested for field work, attendance at meetings and conferences, and other travel associated with the proposed work, including subsistence. To qualify for support, however, attendance at meetings or conferences must be necessary to accomplish proposal objectives or disseminate research results. Travel support for dependents of key project personnel may be requested only when the travel is for a duration of six months or more either by inclusion in the approved budget or with the prior written approval of the cognizant NSF Grants and Agreements Officer. Temporary dependent care costs above and beyond regular dependent care that directly result from travel to conferences are allowable costs provided that the conditions established in 2 CFR §200.475 are met.

Allowance for air travel normally will not exceed the cost of round-trip, economy airfares. Persons traveling under NSF awards must travel by U.S.-Flag Air carriers, if available.

(b) Domestic Travel
Domestic travel includes travel within and between the U.S., its territories, and possessions.[16] Travel, meal, and hotel expenses of recipient employees who are not on travel status are unallowable. Costs of employees on travel status are limited to those specifically authorized by 2 CFR §200.475.

(c) Foreign Travel
Travel outside the areas specified above is considered foreign travel. When anticipated, the proposer must enter the names of countries and dates of visit on the proposal budget, if known.

(v) Participant Support (Line F on the Proposal Budget)
This budget category refers to direct costs for items such as stipends or subsistence allowances, travel allowances, and registration fees paid to or on behalf of participants or trainees (but not employees) in connection with NSF-sponsored conferences or training projects. Any additional categories of participant support costs other than those described in 2 CFR §200.1 (such as incentives, gifts, souvenirs, t-shirts, and memorabilia), must be justified in the budget justification, and such costs will be closely scrutinized by NSF. (See also Chapter II.F.7.) Speakers and trainers generally are not considered participants and should not be included in this section of the budget. However, if the primary purpose of the individual’s attendance at the conference is learning and receiving training as a participant, then the costs may be included under participant support. If the primary purpose is to speak or assist with management of the conference, then such costs should be budgeted in appropriate categories other than participant support.

For some educational projects conducted at local school districts, the participants being trained are employees. In such cases, the costs must be classified as participant support if payment is made through a stipend or training allowance method. The school district must have an accounting mechanism in place (i.e., sub-account code) to differentiate between regular salary and stipend payments.

To help defray the costs of participating in a conference or training activity, funds may be proposed for payment of stipends, per diem or subsistence allowances, based on the type and duration of the activity. Such allowances must be reasonable, in conformance with the policy of the proposing organization and limited to the days of attendance at the conference plus the actual travel time required to reach the conference location. Where meals or lodgings are furnished without charge or at a nominal cost (e.g., as part of the registration fee), the per diem or subsistence allowance should be correspondingly reduced. Although local participants may participate in conference meals and coffee breaks, funds may not be proposed to pay per diem or similar expenses for local participants in the conference. Costs related to an NSF-sponsored conference (e.g., venue rental fees, catering costs, supplies, etc.) that will be secured through a service agreement/contract should be budgeted on line G.6., “Other Direct Costs” to ensure appropriate allocation of indirect costs.

This section of the budget also may not be used for incentive payments to research subjects. Human subject payments should be included on line G.6. of the NSF budget under “Other Direct Costs,” and any applicable indirect costs should be calculated on the payments in accordance with the organization’s Federally negotiated indirect cost rate.

Funds may be requested for the travel costs of participants. If so, the restrictions regarding class of accommodations and use of U.S.-Flag air carriers are applicable.[17] In training activities that involve off-site field work, costs of transportation of participants are allowable. The number of participants to be supported must be entered in the parentheses on the proposal budget. Participant support costs must be specified, itemized, and justified in the budget justification section of the proposal. Indirect costs (F&A) are not usually allowed on costs budgeted as participant support unless the recipient’s current, Federally approved indirect cost rate agreement provides for allocation of F&A to participant support costs. Participant support costs must be accounted for separately should an award be made.

(vi) Other Direct Costs (Lines G1 through G6 on the Proposal Budget)
Any costs proposed to an NSF project must be allowable, reasonable, and directly allocable to the supported activity. When anticipated, the budget must identify and itemize other anticipated direct costs not included under the headings above, including materials and supplies, publication costs, and computer and vendor services. Examples include aircraft rental, space rental at research establishments away from the proposing organization, minor building alterations, payments to human subjects, and service charges. Reference books and periodicals only may be included on the proposal budget if they are specifically allocable to the project being supported by NSF.

(a) Materials and Supplies (including Costs of Computing Devices) (Line G1 on the Proposal Budget)
When anticipated, the proposal budget justification must indicate the general types of expendable materials and supplies required. Supplies are defined as all tangible personal property other than those described in paragraph (d)(iii) above. A computing device is considered a supply if the acquisition cost is less than the lesser of the capitalization level established by the proposer or $5,000, regardless of the length of its useful life. In the specific case of computing devices, charging as a direct cost is allowable for devices that are essential and allocable, but not solely dedicated, to the performance of the NSF project. Details and justification must be included for items requested to support the project.

(b) Publication/Documentation/Dissemination (Line G2 on the Proposal Budget)
The proposal budget may request funds for the costs of documenting, preparing, publishing or otherwise making available to others the findings and products of the work to be conducted under the award. This generally includes the following types of activities: reports, reprints, page charges or other journal costs (except costs for prior or early publication); necessary illustrations; cleanup, documentation, storage and indexing of data and databases; development, documentation and debugging of software; and storage, preservation, documentation, indexing, etc., of physical specimens, collections, or fabricated items. Line G.2. of the proposal budget also may be used to request funding for data deposit and data curation costs.[18]

(c) Consultant Services (also referred to as Professional Service Costs) (Line G3 on the Proposal Budget)[19]
The proposal budget may request costs for professional and consultant services. Professional and consultant services are services rendered by persons who are members of a particular profession or possess a special skill, and who are not officers or employees of the proposing organization. Costs of professional and consultant services are allowable when reasonable in relation to the services rendered and when not contingent upon recovery of costs from the Federal government. Anticipated services must be justified, and information furnished on each individual’s expertise, primary organizational affiliation, normal daily compensation rate, and number of days of expected service. Consultants’ travel costs, including subsistence, may be included. If requested, the proposer must be able to justify that the proposed rate of pay is reasonable. Additional information on the allowability of consultant or professional service costs is available in 2 CFR §200.459. In addition to other provisions required by the proposing organization, all contracts made under the NSF award must contain the applicable provisions identified in 2 CFR §200 Appendix II.

(d) Computer Services (Line G4 on the Proposal Budget)
The cost of computer services, including computer-based retrieval of scientific, technical, and educational information, may be requested only where it is institutional policy to charge such costs as direct charges. A justification based on the established computer service rates at the proposing organization must be included. The proposal budget also may request costs for leasing of computer equipment.

(e) Subawards (Line G5 on the Proposal Budget)[20] [21]
Except for the purpose of obtaining goods and services for the proposer's own use which creates a procurement relationship with a contractor, no portion of the proposed activity may be subawarded or transferred to another organization without prior written NSF authorization. Such authorization must be provided either through approval specifically granted in the award notice or by receiving written prior approval from the cognizant NSF Grants and Agreements Officer after an award is issued. If the subaward organization is changed, prior approval of the new subaward organization must be obtained from the cognizant NSF Grants and Agreements Officer.

If known at the time of proposal submission, the intent to enter into such arrangements must be disclosed in the proposal. A separate budget and a budget justification of no more than five pages, must be provided for each subrecipient, if already identified. The description of the work to be performed must be included in the project description.

All proposing organizations are required to make a case-by-case determination regarding the role of a subrecipient versus contractor for each agreement it makes. 2 CFR §200.331 provides characteristics of each type of arrangement to assist proposing organizations in making that determination. However, inclusion of a subaward or contract in the proposal budget or submission of a request after issuance of an NSF award to add a subaward or contract will document the required organizational determination.

NSF does not negotiate rates for organizations that are not direct recipients of NSF funding (e.g., subrecipients). Consistent with 2 CFR §200.332, NSF recipients must use the domestic subrecipient’s applicable U.S. Federally negotiated indirect cost rate(s). If no such rate exists, the NSF recipient must determine the appropriate rate in collaboration with the subrecipient. The appropriate rate will be: a negotiated rate between the NSF recipient and the subrecipient; a prior rate negotiated between a different pass-through entity and the same subrecipient, or the de minimis indirect cost recovery rate of 10% of modified total direct costs.

It is also NSF’s expectation that NSF recipients will use foreign subrecipients’ applicable U.S. Federally negotiated indirect cost rate(s). However, if no such rate exists, the NSF recipient will fund the foreign subrecipient using the de minimis indirect cost rate recovery of 10% of modified total direct costs. See also Chapter I.E.2. for additional requirements on issuance of a subaward or consultant arrangement to a foreign organization or a foreign individual.

Proposers are responsible for ensuring that proposed subrecipient costs, including indirect costs, are reasonable and appropriate.

(f) Other (Line G6 on the Proposal Budget)[22]
Any other direct costs not specified in Lines G.1. through G.5. must be identified on Line G.6. Such costs must be itemized and detailed in the budget justification. Examples include:

Contracts for the purpose of obtaining goods and services for the proposer’s own use (see 2 CFR §200.331 for additional information); and
Incentive payments, for example, payments to human subjects or incentives to promote completion of a survey, should be included on line G.6. of the NSF budget. Incentive payments should be proposed in accordance with organizational policies and procedures. Indirect costs should be calculated on incentive payments in accordance with the organization’s approved U.S. Federally negotiated indirect cost rate(s). Performance based incentive payments to employees as described in 2 CFR §200.430(f) should not be included in this section of the budget.
(vii) Total Direct Costs (Line H on the Proposal Budget)
The total amount of direct costs requested in the budget, to include Lines A through G, must be entered on Line H.

(viii) Indirect Costs (also known as Facilities and Administrative Costs (F&A) for Colleges and Universities) (Line I on the Proposal Budget)
Except where specifically identified in an NSF program solicitation, the applicable U.S. Federally negotiated indirect cost rate(s) must be used in computing indirect costs (F&A) for a proposal. Use of an indirect cost rate lower than the organization’s current negotiated indirect cost rate is considered a violation of NSF’s cost sharing policy. See section (xii) below. The amount for indirect costs must be calculated by applying the current negotiated indirect cost rate(s) to the approved base(s), and such amounts must be specified in the budget justification. Indirect cost recovery for IHEs is additionally restricted by 2 CFR §200, Appendix III, paragraph C.7. which specifies Federal agencies are required to use the negotiated F&A rate that is in effect at the time of the initial award throughout the life of the sponsored agreement. Additional information on the charging of indirect costs to an NSF award is available in Chapter X.D.

Domestic proposing organizations that do not have a current negotiated rate agreement with a cognizant Federal agency may choose to apply the de minimis rate of 10% to a base of modified total direct costs (MTDC) as authorized by 2 CFR §200.414(f). No supporting documentation is required for proposed rates of 10% or less of modified total direct costs. Organizations without a current negotiated indirect cost rate agreement and that wish to request indirect cost rate recovery above 10% should prepare an indirect cost proposal based on expenditures for its most recently ended fiscal year. Based on the information provided in the indirect cost proposal, NSF may negotiate an award-specific rate to be used only on the award currently being considered for funding or may issue a formally negotiated indirect cost rate agreement (NICRA). The contents and financial data included in indirect cost proposals vary according to the make-up of the proposing organization. Instructions for preparing an indirect cost rate proposal can be found on the NSF website. NSF formally negotiates indirect cost rates for the organizations for which NSF has rate cognizance. NSF does not negotiate rates for entities that do not yet hold direct NSF funding, nor does NSF negotiate rates for subrecipients.

Foreign organizations that do not have a current U.S. Federally negotiated indirect cost rate(s) are limited to a de minimis indirect cost rate recovery of 10% of modified total direct costs. Foreign recipients that have a U.S. Federally negotiated indirect cost rate(s) may recover indirect costs at the current negotiated rate.

(ix) Total Direct and Indirect Costs (F&A) (Line J on the Proposal Budget)
The total amount of direct and indirect costs (F&A) (sum of Lines H and I) must be entered on Line J.

(x) Fees (Line K on the Proposal Budget)
This line is available for use only by the SBIR/STTR programs and Major Facilities programs when specified in the solicitation.

(xi) Amount of This Request (Line L on the Proposal Budget)
The total amount of funds requested by the proposer.

(xii) Cost Sharing (Line M on the Proposal Budget)
The National Science Board issued a report entitled "Investing in the Future: NSF Cost Sharing Policies for a Robust Federal Research Enterprise" (NSB 09-20, August 3, 2009), which contained eight recommendations for NSF regarding cost sharing. In implementation of the Board’s recommendation, NSF’s guidance[23] is as follows:

Voluntary Committed and Uncommitted Cost Sharing
As stipulated in 2 CFR §200.1, Voluntary committed cost sharing means cost sharing specifically pledged on a voluntary basis in the proposal's budget or the Federal award on the part of the non-Federal entity and that becomes a binding requirement of Federal award.” As such, to be considered voluntary committed cost sharing, the amount must appear on the NSF proposal budget and be specifically identified in the approved NSF budget.[24] Unless required by NSF (see the section on Mandatory Cost Sharing below), inclusion of voluntary committed cost sharing is prohibited and Line M on the proposal budget will not be available for use by the proposer. NSF Program Officers are not authorized to impose or encourage mandatory cost sharing unless such requirements are explicitly included in the program solicitation.

In order for NSF, and its reviewers, to assess the scope of a proposed project, all organizational resources necessary for, and available to, a project must be described in the Facilities, Equipment and Other Resources section of the proposal (see Chapter II.D.2.g for further information). While not required by NSF, the recipient may, at its own discretion, continue to contribute voluntary uncommitted cost sharing to NSF-sponsored projects. As noted above, however, these resources are not auditable by NSF and should not be included in the proposal budget or budget justification.

Mandatory Cost Sharing
Mandatory cost sharing will only be required for NSF programs when explicitly authorized by the NSF Director, the NSB, or legislation. A complete listing of NSF programs that require cost sharing is available on the NSF website. In these programs, cost sharing requirements will be clearly identified in the solicitation and must be included on Line M of the proposal budget. For purposes of budget preparation, the cumulative cost sharing amount must be entered on Line M of the first year’s budget. Should an award be made, the organization’s cost sharing commitment, as specified on the first year’s approved budget, must be met prior to the award end date.

Such cost sharing will be considered as an eligibility, rather than a review criterion. Proposers are advised not to exceed the mandatory cost sharing level or amount specified in the solicitation.[25]

When mandatory cost sharing is included on Line M, and accepted by the Foundation, the commitment of funds becomes legally binding and is subject to audit. When applicable, the estimated value of any in-kind contributions also should be included on Line M. An explanation of the source, nature, amount, and availability of any proposed cost sharing must be provided in the budget justification[26]. It should be noted that contributions derived from other Federal funds or counted as cost sharing toward projects of another Federal agency must not be counted towards meeting the specific cost sharing requirements of the NSF award.

Failure to provide the level of cost sharing required by the NSF solicitation and reflected in the NSF award budget may result in termination of the NSF award, disallowance of award costs and/or refund of award funds to NSF by the recipient.

(xiii) Allowable and Unallowable Costs
2 CFR §200, Subpart E provides comprehensive information regarding costs allowable under Federal awards. The following categories of unallowable costs are highlighted because of their sensitivity:

(a) Entertainment
Costs of entertainment, amusement, diversion and social activities, and any costs directly associated with such activities (such as tickets to shows or sporting events, meals, lodging, rentals, transportation, and gratuities) are unallowable. When costs typically considered as entertainment are necessary to accomplish the proposed objectives, they must be included in the budget and justified in the budget justification. Travel, meal, and hotel expenses of recipient employees who are not on travel status are unallowable. See also 2 CFR §200.438.

(b) Meals and Coffee Breaks
No funds may be requested for meals or coffee breaks for intramural meetings of an organization or any of its components, including, but not limited to, laboratories, departments, and centers. (See 2 CFR §200.432, for additional information on the charging of certain types of costs generally associated with conferences supported by NSF.) Meal expenses of recipient employees who are not on travel status are unallowable. See also 2 CFR §200.438.

(c) Alcoholic Beverages
No NSF funds may be requested or spent for alcoholic beverages.

(d) Home Office Workspace
Rental of any property owned by individuals or entities affiliated with NSF recipients (including commercial or residential real estate), for use as home office workspace is unallowable. See 2 CFR §200.465(f).

(e) Prohibition on Certain Telecommunications and Video Surveillance Services or Equipment
Section 889 of the National Defense Authorization Act (NDAA) for Fiscal Year (FY) 2019 (Public Law 115-232) prohibits the head of an executive agency from obligating or expending loan or grant funds to procure or obtain, extend, or renew a contract to procure or obtain, or enter into a contract (or extend or renew a contract) to procure or obtain the equipment, services, or systems as identified in section 889 of the NDAA for FY 2019. See 2 CFR §§200.216, 200.471, PAPPG Chapter X.F, and the applicable award terms and conditions for additional information
}
