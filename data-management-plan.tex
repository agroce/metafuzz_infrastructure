 %%%%%%%%%%%%%%%%%%%%%%%%%%%%%%%%%%%%%%%%%%%%%%%%%%%%%%%%%%%%%
\documentclass[12pt]{article}

\usepackage{fullpage}

\begin{document}

\pagenumbering{gobble}

\begin{center}
  {\Large\sf\textbf{Collaborative Research: CIRC: New: Medium: A Development and Experimental Environment for Meta-Fuzzing}}
\end{center}

The data in the proposed project is primarily source code for The
source code will be released either in a project GitHub repository (or
respositories) to be created. Any experimental data should be reproducible using the code artifacts, but will also be published in a condensed form in the repositories.

Curricular
materials and documentation associated with any tools will also be stored in
an open source repository, since in this project the primary
educational benefits are linked to the use of these tools.  Using GitHub automatically provides us with excellent backup
and archiving for the code and curricular material products of the
project.

We have no unusual format or metadata requirements; the
tools involved primarily use textual formats that are easy to store
and read or byte-based formats that are interpreted only by fuzzer
drivers or code under test.

\end{document}
